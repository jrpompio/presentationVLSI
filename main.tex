%% presentacion.tex
\documentclass[aspectratio=169]{beamer}

\usetheme[]{lu}
% \setlufootleft{Autor — Curso/Institución}

\usepackage{appendixnumberbeamer}
\graphicspath{{graphics/}}

\title{DREAMPlace\\Placement analítico acelerado en GPU}
\author{Trabajo basado en el paper suministrado}
\institute{Presentación académica}

\begin{document}

\begin{frame}[plain]
  \titlepage
\end{frame}

\begin{frame}{Objetivo y promesa central}
  \begin{itemize}
    \item Acelerar el \emph{global placement} mediante GPU manteniendo
          calidad.
    \item Integración con toolkits de DL para optimización eficiente.
    \item Resultados reportados: mejoras de tiempo de orden decenas de
          veces.
  \end{itemize}
\end{frame}

\begin{frame}{Estado del arte y motivación}
  \begin{itemize}
    \item RePlAce en CPU como línea base ampliamente usada.
    \item Saturación del rendimiento en CPU al escalar el tamaño del
          diseño.
    \item Hipótesis: reutilizar kernels GPU y autograd para acelerar sin
          degradar métricas.
  \end{itemize}
\end{frame}

\begin{frame}{Analogía con aprendizaje profundo}
  \begin{columns}[onlytextwidth]
    \begin{column}{.55\textwidth}
      \begin{itemize}
        \item Función de costo $\rightarrow$ wirelength suavizado.
        \item Regularización $\rightarrow$ penalización de densidad.
        \item Optimización de primer orden con \emph{autograd}.
        \item Bucle iterativo tipo entrenamiento: forward/backward.
      \end{itemize}
    \end{column}
    \begin{column}{.45\textwidth}
      \begin{figure}
        \centering
        % \includegraphics[width=\linewidth]{fig1_analogia}
        \includegraphics[width=\linewidth]{example.jpeg}
        \caption{Analogía DL $\leftrightarrow$ placement.}
      \end{figure}
    \end{column}
  \end{columns}
\end{frame}

\begin{frame}{Arquitectura y flujo}
  \begin{itemize}
    \item Núcleo en C++/CUDA expuesto a Python mediante el framework DL.
    \item Operadores para wirelength, densidad y gradientes.
    \item Flujo: Global Placement (GPU) $\rightarrow$ Legalización
          (CPU) $\rightarrow$ Detailed Placement.
  \end{itemize}
  \begin{figure}
    \centering
    % \includegraphics[width=.9\linewidth]{fig2_arquitectura_flujo}
        \includegraphics[width=\linewidth]{example.jpeg}
    \caption{Arquitectura y flujo general GP$\rightarrow$LG$\rightarrow$DP.}
  \end{figure}
\end{frame}

\begin{frame}{Núcleo algorítmico (visión intuitiva)}
  \begin{columns}[onlytextwidth]
    \begin{column}{.55\textwidth}
      \begin{itemize}
        \item Wirelength suavizado: aproximación diferenciable de HPWL.
        \item Densidad por analogía electrostática y solución de Poisson.
        \item Uso de DCT/IDCT para resolver el potencial de forma
              eficiente.
        \item Gradientes exactos producidos por \emph{autograd}+kernels.
      \end{itemize}
    \end{column}
    \begin{column}{.45\textwidth}
      \begin{figure}
        \centering
        % \includegraphics[width=\linewidth]{fig4_pipeline_densidad}
        \includegraphics[width=\linewidth]{example.jpeg}
        \caption{Esquema de la penalización de densidad.}
      \end{figure}
    \end{column}
  \end{columns}
\end{frame}

\begin{frame}{Aceleraciones en GPU (kernels)}
  \begin{itemize}
    \item Kernel de wirelength a nivel de pin con operaciones atómicas.
    \item DCT/IDCT implementadas con tamaños óptimos para el dominio.
    \item Minimizar transferencias CPU$\leftrightarrow$GPU en el bucle
          de optimización.
  \end{itemize}
  \begin{figure}
    \centering
    % \includegraphics[width=.9\linewidth]{fig7y8_kernels_gpu}
        \includegraphics[width=\linewidth]{example.jpeg}
    \caption{Comparativas de kernels de wirelength y DCT/IDCT.}
  \end{figure}
\end{frame}

\begin{frame}{Resultados: velocidad y escala}
  \begin{columns}[onlytextwidth]
    \begin{column}{.55\textwidth}
      \begin{itemize}
        \item Aceleraciones de orden $\times$30 o más en \emph{global
              placement}.
        \item Escalado estable al aumentar el número de celdas del
              diseño.
        \item Calidad del placement comparable frente al baseline.
      \end{itemize}
    \end{column}
    \begin{column}{.45\textwidth}
      \begin{figure}
        \centering
        % \includegraphics[width=\linewidth]{fig5_escalado}
        \includegraphics[width=\linewidth]{example.jpeg}
        \caption{Escalado en tiempo de ejecución con el tamaño.}
      \end{figure}
    \end{column}
  \end{columns}
\end{frame}

\begin{frame}{Desglose de tiempo y cuellos de botella}
  \begin{itemize}
    \item GP y legalización consumen una fracción del tiempo total.
    \item Detailed Placement domina el tiempo total del flujo.
    \item Dentro de GP, la penalización de densidad es el componente más
          costoso.
  \end{itemize}
  \begin{figure}
    \centering
    % \includegraphics[width=.9\linewidth]{fig6_breakdown}
        \includegraphics[width=\linewidth]{example.jpeg}
    \caption{Distribución de tiempo por etapa y por operador.}
  \end{figure}
\end{frame}

\begin{frame}{Conclusiones y trabajo futuro}
  \begin{itemize}
    \item Marco extensible: Python con operadores C++/CUDA especializados.
    \item Aceleración significativa en GP sin degradar métricas de
          calidad.
    \item Trabajo futuro: DP en GPU, multi-GPU, rutabilidad y temporizado
          en bucle.
  \end{itemize}
\end{frame}

\begin{frame}[plain]
  \Closure{Gracias}{¿Preguntas?}
\end{frame}

\end{document}

